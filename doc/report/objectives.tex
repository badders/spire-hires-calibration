\subsection{Aims and Objectives}

The primary aim of the project is to assist the SPIRE Post-Operations Team in the evaluation of the HIRES routine to enhance angular resolution of SPIRE maps. A procedure for evaluating the effectiveness of HiRes on an observation and set of criteria for establishing whether to use HiRes on an observation should be used needs to be established before September 2015.

This will require analyzing the output of HiRes and establishing a metric for determining the level of improvement (if any) provided on an existing observation.

The method proposed for this is to generate simulated observational data from a higher resolution source, and then use this as a basis of comparison for testing the effectiveness of HiRes, focusing on how the signal to noise of an observation effects the fidelity of HiRes images.

It is proposed that examining the power spectra of both the simulated observations and the output of HiRes, and comparing these to the power spectra of the higher resolution source material may reveal useful information. Additionally the extra information returned by the HiRes routine may hold details about effectiveness. Difference maps of the images produced may also hold other information, and most importantly reveal if any additional artifacts are created.
